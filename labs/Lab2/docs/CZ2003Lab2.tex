
\documentclass[acmlarge,nonacm=true]{acmart}
\usepackage{adjustbox}
\usepackage{multirow}
\usepackage{graphicx}
\usepackage{afterpage}
\usepackage{subcaption}

\newcommand\blankpage{%
	\null
	\thispagestyle{empty}%
	\addtocounter{page}{-1}%
	\newpage}

%%
%% \BibTeX command to typeset BibTeX logo in the docs
\AtBeginDocument{%
  \providecommand\BibTeX{{%
    \normalfont B\kern-0.5em{\scshape i\kern-0.25em b}\kern-0.8em\TeX}}}


\begin{document}
	
	\begin{titlepage}
		\begin{center}
			\vspace*{1cm}
			\includegraphics[width=0.7\textwidth]{fig/ntu_logo}
			\vspace{0.8cm}
			\linebreak
			\Huge
			\textbf{Experiment 1: Parametric Curves}
			
			\vspace{0.5cm}
			\LARGE
			CZ2003 Computer Graphics and Visualization
			
			\vspace{1.5cm}
			\textbf{SS3}\\
			
			\begin{table}[h]
				\begin{tabular}{lc}
					Name & Matric Number \\\hline
					Pang Yu Shao & U17216\underline{\textbf{80}}D \\
				\end{tabular}
			\end{table}
			
			
			
			\vfill
			
			\vspace{0.8cm}
			
			
			
			\Large
			SCHOOL OF COMPUTER SCIENCE AND ENGINEERING\\
			NANYANG TECHNOLOGICAL UNIVERSITY\\
			SINGAPORE\\
			16th Febraury 2021
			
		\end{center}
	\end{titlepage}

 

%%
%% The "title" command has an optional parameter,
%% allowing the author to define a "short title" to be used in page headers.
\title{CZ2003 Computer Graphics and Visualization}

%%
%% The "author" command and its associated commands are used to define
%% the authors and their affiliations.
%% Of note is the shared affiliation of the first two authors, and the
%% "authornote" and "authornotemark" commands
%% used to denote shared contribution to the research.


\author{Pang Yu Shao}
\email{C170134@e.ntu.edu.sg}
\affiliation{\institution{Nanyang Technological University}}

%%
%% By default, the full list of authors will be used in the page
%% headers. Often, this list is too long, and will overlap
%% other information printed in the page headers. This command allows
%% the author to define a more concise list
%% of authors' names for this purpose.
\renewcommand{\shortauthors}{Pang Yu Shao}






%%
%% This command processes the author and affiliation and title
%% information and builds the first part of the formatted document.

% \begin{teaserfigure}
% 	\includegraphics[width=\textwidth]{bccell}
% 	\caption{Breast Cancer Cell. Photograph by National Cancer Institute [Public domain], via Wikimedia
% 		Commons. (\url{https://w.wiki/kS3}).}
% 	\Description{A breast cancer cell seen through an electron microscope.}
% \end{teaserfigure}
% \maketitle



\tableofcontents
\newpage
\section{Defining Surfaces Parametrically}
\subsection{Plane Passing Through Three Defined Points}
To define the plane parametrically, we can use the following formula: \(P = P1+u(P2-P1)+v(P3-P1)\)\\
Therefore, with the 3 points \((N, M, 0),\ (0, M, N),\ (N, 0, M)\), we get:\\
\(x(u,v) = N - Nu = \mathbf{8 - 8u}\)\\
\(y(u,v) = M + Mv = \mathbf{10 + 10v}\)\\
\(z(u,v) = Nu + Mv = \mathbf{8u + 10v}\)\\
\(\mathbf{u,v \in [0,1]}\)

\begin{figure}[H]
	\begin{subfigure}{.33\textwidth}
	  \centering
	  \includegraphics[width=.8\linewidth]{fig/1a1_1}
	  \caption{Resolution: 1 1}
	\end{subfigure}%
	\begin{subfigure}{.33\textwidth}
	  \centering
	  \includegraphics[width=.8\linewidth]{fig/1a10_10}
	  \caption{Resolution: 10 10}
	\end{subfigure}
	\begin{subfigure}{.33\textwidth}
		\centering
		\includegraphics[width=.8\linewidth]{fig/1a1000_1000}
		\caption{Resolution: 1000 1000}
	  \end{subfigure}
	\caption{Plots of the plane defined in "\textbf{1a.wrl}" with differing resolutions}
	\label{fig:1a}
\end{figure}
As seen in Fig. \ref{fig:1a} above, a sampling resolution of \textbf{1} for both u and v is 
sufficient for drawing the plane as it has no curvature and having a higher resolution would
produce the exact same drawing.

\subsection{Triangular Polygon with Three Defined Vertices}
To define the Triangular Polygon, we use the formula for defining Bilinear Surface Parametrically,
and we set two of the points to be the same point, essentially resulting in a Triangular polygon.\\\\
\(P = P1 + u(P2-P1) + v(P3-P1+u(P4-P3-(P2-P1)))\)\\
Let P4 = P3, we get:\\
\(P = P1 + u(P2 - P1) + v(P3 - P1) + uv(P1 - P2)\)\\
Therefore, with the 3 points \((N, M, 0),\ (0, M, N),\ (N, 0, M)\), we get:\\
\(x(u,v) = N - Nu + Nuv = \mathbf{8 - 8u + 8uv}\)\\
\(y(u,v) = M - Mv = \mathbf{10 - 10v}\)\\
\(z(u,v) = Nu + Mv - Nuv = \mathbf{8u + 10v - 8uv}\)\\
\(\mathbf{u,v \in [0,1]}\)

\begin{figure}[H]
	\begin{subfigure}{.33\textwidth}
	  \centering
	  \includegraphics[width=.8\linewidth]{fig/1b1_1}
	  \caption{Resolution: 1 1}
	\end{subfigure}%
	\begin{subfigure}{.33\textwidth}
	  \centering
	  \includegraphics[width=.8\linewidth]{fig/1b10_10}
	  \caption{Resolution: 10 10}
	\end{subfigure}
	\begin{subfigure}{.33\textwidth}
		\centering
		\includegraphics[width=.8\linewidth]{fig/1b1000_1000}
		\caption{Resolution: 1000 1000}
	  \end{subfigure}
	\caption{Plots of the Triangular Polygon defined in "\textbf{1b.wrl}" with differing resolutions}
	\label{fig:1b}
\end{figure}

As seen in Fig. \ref{fig:1b} above, a sampling resolution of \textbf{1} for both u and v is 
sufficient for drawing the triangular polygon as it has no curvature and having a higher resolution would
produce the exact same drawing.


\subsection{Origin-Centered Ellipsold with Defined Semi-axes}
An elipsoid can be parametrically defined using the following functions:\\
\(x = acos(u)sin(v)\)\\
\(y = bsin(u)\)\\
\(z = ccos(u)cos(v)\)\\
Where \(u \in [-\pi/2, \pi/2]\) and \(v \in [-\pi, \pi]\). Therefore, we can get:\\
\(x = Ncos(-\pi/2 + \pi u)sin(-\pi + 2\pi v) = \mathbf{8cos(-\pi/2 + \pi u)sin(-\pi + 2\pi v)}\)\\
\(y = Msin(-\pi/2 + \pi u) = \mathbf{10sin(-\pi/2 + \pi u)}\)\\
\(z = (N+M/2)cos(-\pi/2 + \pi u)cos(-\pi + 2\pi v) = \mathbf{9cos(-\pi/2 + \pi u)cos(-\pi + 2\pi v)}\)\\
\(u,v \in [0,1]\)

\begin{figure}[H]
	\begin{subfigure}{.33\textwidth}
	  \centering
	  \includegraphics[width=.8\linewidth]{fig/1c50_100}
	  \caption{Resolution: 50 100}
	\end{subfigure}%
	\begin{subfigure}{.33\textwidth}
	  \centering
	  \includegraphics[width=.8\linewidth]{fig/1c25_50}
	  \caption{Resolution: 25 50}
	\end{subfigure}
	\begin{subfigure}{.33\textwidth}
		\centering
		\includegraphics[width=.8\linewidth]{fig/1c100_200}
		\caption{Resolution: 100 200}
	  \end{subfigure}
	\caption{Plots of the Ellipsold defined in "\textbf{1c.wrl}" with differing resolutions}
	\label{fig:1c}
\end{figure}

The sampling resolution for v is chosen to be 2 times larger than that of u as the coefficient of v is 
2 times larger. As seen in Fig. \ref{fig:1c} above, the best resolution obtained was \textbf{50, 100} for u and v respectively.
By decreasing the resolution to 25 and 50, the interpolations could be seen on the curvature when zoomed in. 
When the resolution was increased instead to 100 and 200, no visible difference could be seen.

\subsection{Cylindrical Surface with Defined Radius}
\label{section:1d}
To define a cylindrical surface parametrically, we can define it by first 
defining a circular curve, followed by performing translational sweeping along the
Z-axis to obtain the cylindrical surface.\\
Therefore we can obtain the following equations and surface:\\
\(x = Ncos(2\pi u) = \mathbf{8cos(2\pi u)}\)\\
\(y = Nsin(2\pi u) = \mathbf{8sin(2\pi u)}\)\\
\(z = -N + (M+N)v = \mathbf{-8 + 18v}\)\\
\(u,v \in [0,1]\)
\begin{figure}[H]
	\begin{subfigure}{.33\textwidth}
	  \centering
	  \includegraphics[width=.8\linewidth]{fig/1d75_1}
	  \caption{Resolution: 75 1}
	\end{subfigure}%
	\begin{subfigure}{.33\textwidth}
	  \centering
	  \includegraphics[width=.8\linewidth]{fig/1d50_1}
	  \caption{Resolution: 50 1}
	\end{subfigure}
	\begin{subfigure}{.33\textwidth}
		\centering
		\includegraphics[width=.8\linewidth]{fig/1d200_1}
		\caption{Resolution: 200 1}
	  \end{subfigure}
	\caption{Plots of the Cylindrical Surface defined in "\textbf{1d.wrl}" with differing resolutions}
	\label{fig:1d}
\end{figure}

The sampling resolution for v is chosen to be 1 as the parameter is responsible for the straight line sweep.
 As seen in Fig. \ref{fig:1d} above, the best resolution obtained was \textbf{75, 1} for u and v respectively.
By decreasing the resolution to 50 for u, the interpolations could be seen on the curvature when zoomed in. When the resolution of u was 
increased instead to 200, no visible difference could be seen between that and when the resolution was 75.

\section{Defining Surface by Translational Sweeping of Curve}
We obtain the equations of the curve from lab 1:

\begin{displaymath}
	\mathbf{x(u) = -10.4 + 26.4*u}
\end{displaymath}
\begin{displaymath}
	\mathbf{y(u) = tanh(-10.4 + 26.4*u)}
\end{displaymath}
\begin{displaymath}
	u \in [0,1]
\end{displaymath}

To obtain a surface by translational sweeping of the curve, we can introduce the v parameter in the function of Z.
Therefore, we get:\\
\begin{displaymath}
	\mathbf{x(u,v) = -10.4 + 26.4*u}
\end{displaymath}
\begin{displaymath}
	\mathbf{y(u,v) = tanh(-10.4 + 26.4*u)}
\end{displaymath}
\begin{displaymath}
	\mathbf{z(u,v) = -N + (N+M)v = -8 + 18v}
\end{displaymath}
\begin{displaymath}
	u,v \in [0,1]
\end{displaymath}

\begin{figure}[H]
	\begin{subfigure}{.33\textwidth}
	  \centering
	  \includegraphics[width=.8\linewidth]{fig/2_200_1}
	  \caption{Resolution: 200 1}
	\end{subfigure}%
	\begin{subfigure}{.33\textwidth}
	  \centering
	  \includegraphics[width=.8\linewidth]{fig/2_100_1}
	  \caption{Resolution: 100 1}
	\end{subfigure}
	\begin{subfigure}{.33\textwidth}
		\centering
		\includegraphics[width=.8\linewidth]{fig/2_1000_1}
		\caption{Resolution: 1000 1}
	  \end{subfigure}
	\caption{Plots of the Surface defined in "\textbf{2.wrl}" with differing resolutions}
	\label{fig:2}
\end{figure}

Similar to the surface defined in section \ref{section:1d}, the sampling resolution for v is chosen to be 1 as the parameter is responsible for the straight line sweep.
 As seen in Fig. \ref{fig:2} above, the best resolution obtained was \textbf{200, 1} for u and v respectively.
By decreasing the resolution to 100 for u, the interpolations could be seen on the curvature when zoomed in. When the resolution of u was 
increased instead to 1000, no visible difference could be seen between that and when the resolution was 200.

\section{Defining Surface by Rotational Sweeping of Curve}
We obtain the equations of the curve in lab 1:
\begin{displaymath}
	\mathbf{x(u) =  (8 - 15cos(2\pi u))cos(2\pi u)}
\end{displaymath}
\begin{displaymath}
	\mathbf{y(u) =  (8 - 15cos(2\pi u))sin(2\pi u)}
\end{displaymath}
\begin{displaymath}
	u \in [0, 1]
\end{displaymath}

After translating by -N in the Z axis, then with clockwise rotational sweeping of \(\pi/N\) performed with an offset angle
of \(+\frac{3\pi}{2M}\), where N = 8 and M = 10, we get the following functions and surface:\\


\begin{displaymath}
	\mathbf{x(u,v) =  ((8 - 15cos(2\pi u))cos(2\pi u)-8)sin(\frac{3\pi}{20} - \frac{\pi}{8}v)}
\end{displaymath}
\begin{displaymath}
	\mathbf{y(u,v) =  (8 - 15cos(2\pi u))sin(2\pi u)}
\end{displaymath}
\begin{displaymath}
	\mathbf{z(u,v) =  ((8 - 15cos(2\pi u))cos(2\pi u)-8)cos(\frac{3\pi}{20} - \frac{\pi}{8}v)}
\end{displaymath}
\begin{displaymath}
	u,v \in [0, 1]
\end{displaymath}

\begin{figure}[H]
	\begin{subfigure}{.33\textwidth}
	  \centering
	  \includegraphics[width=.8\linewidth]{fig/3_200_14}
	  \caption{Resolution: 200 14}
	\end{subfigure}%
	\begin{subfigure}{.33\textwidth}
	  \centering
	  \includegraphics[width=.8\linewidth]{fig/3_100_7}
	  \caption{Resolution: 100 7}
	\end{subfigure}
	\begin{subfigure}{.33\textwidth}
		\centering
		\includegraphics[width=.8\linewidth]{fig/3_400_28}
		\caption{Resolution: 400 28}
	  \end{subfigure}
	\caption{Plots of the Surface defined in "\textbf{3.wrl}" with differing resolutions}
	\label{fig:3}
\end{figure}

The sampling resolution for v is chosen to be approximately 16 times smaller than that of u as the coefficient of v is 
16 times smaller. As seen in Fig. \ref{fig:3} above, the best resolution obtained was \textbf{200, 14} for u and v respectively.
By decreasing the resolution to 100 and 7, the interpolations could be seen on the curvature when zoomed in. When the resolution was increased
instead to 400 and 28, no visible difference could be seen when compared to the shape defined with the resolutions 200 and 14.

\bibliographystyle{ACM-Reference-Format}
\newpage






\end{document}
\endinput
%%
%% End of file `sample-acmlarge.tex'.
