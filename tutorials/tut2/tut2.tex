\documentclass{article}

\usepackage{fancyhdr}
\usepackage{extramarks}
\usepackage{amsmath}
\usepackage{amsthm}
\usepackage{amsfonts}
\usepackage{tikz}
\usepackage[plain]{algorithm}
\usepackage{algpseudocode}
\usepackage{enumerate}
\usepackage{amssymb}

\usetikzlibrary{automata,positioning}

%
% Basic Document Settings
%

\topmargin=-0.45in
\evensidemargin=0in
\oddsidemargin=0in
\textwidth=6.5in
\textheight=9.0in
\headsep=0.25in

\linespread{1.1}

\pagestyle{fancy}
\lhead{\hmwkAuthorName}
\chead{\hmwkClass\ (\hmwkClassInstructor\ \hmwkClassTime): \hmwkTitle}
\rhead{\firstxmark}
\lfoot{\lastxmark}
\cfoot{\thepage}

\renewcommand\headrulewidth{0.4pt}
\renewcommand\footrulewidth{0.4pt}

\setlength\parindent{0pt}

%
% Create Problem Sections
%

\newcommand{\enterProblemHeader}[1]{
    \nobreak\extramarks{}{Problem \arabic{#1} continued on next page\ldots}\nobreak{}
    \nobreak\extramarks{Problem \arabic{#1} (continued)}{Problem \arabic{#1} continued on next page\ldots}\nobreak{}
}

\newcommand{\exitProblemHeader}[1]{
    \nobreak\extramarks{Problem \arabic{#1} (continued)}{Problem \arabic{#1} continued on next page\ldots}\nobreak{}
    \stepcounter{#1}
    \nobreak\extramarks{Problem \arabic{#1}}{}\nobreak{}
}

\setcounter{secnumdepth}{0}
\newcounter{partCounter}
\newcounter{homeworkProblemCounter}
\setcounter{homeworkProblemCounter}{1}
\nobreak\extramarks{Problem \arabic{homeworkProblemCounter}}{}\nobreak{}

%
% Homework Problem Environment
%
% This environment takes an optional argument. When given, it will adjust the
% problem counter. This is useful for when the problems given for your
% assignment aren't sequential. See the last 3 problems of this template for an
% example.
%
\newenvironment{homeworkProblem}[1][-1]{
    \ifnum#1>0
        \setcounter{homeworkProblemCounter}{#1}
    \fi
    \section{Problem \arabic{homeworkProblemCounter}}
    \setcounter{partCounter}{1}
    \enterProblemHeader{homeworkProblemCounter}
}{
    \exitProblemHeader{homeworkProblemCounter}
}

%
% Homework Details
%   - Title
%   - Due date
%   - Class
%   - Section/Time
%   - Instructor
%   - Author
%

\newcommand{\hmwkTitle}{Tutorial 2}
\newcommand{\hmwkDueDate}{January 26, 2021}
\newcommand{\hmwkClass}{CZ2003}
\newcommand{\hmwkClassTime}{SS3}
\newcommand{\hmwkClassInstructor}{Assoc Prof Alexei Sourin}
\newcommand{\hmwkAuthorName}{\textbf{Pang Yu Shao}}
\newcommand{\hmwkAuthorID}{\textbf{U1721680D}}

%
% Title Page
%

\title{
    \vspace{2in}
    \textmd{\textbf{\hmwkClass:\ \hmwkTitle}}\\
    \normalsize\vspace{0.1in}\small{Due\ on\ \hmwkDueDate\ at 10:30am}\\
    \vspace{0.1in}\large{\textit{\hmwkClassInstructor\ - \hmwkClassTime}}
    \vspace{3in}\\
    \hmwkAuthorName\\
    \hmwkAuthorID
}

\date{24/01/2021}

\renewcommand{\part}[1]{\textbf{\large Part \Alph{partCounter}}\stepcounter{partCounter}\\}

%
% Various Helper Commands
%

% Useful for algorithms
\newcommand{\alg}[1]{\textsc{\bfseries \footnotesize #1}}

% For derivatives
\newcommand{\deriv}[1]{\frac{\mathrm{d}}{\mathrm{d}x} (#1)}

% For partial derivatives
\newcommand{\pderiv}[2]{\frac{\partial}{\partial #1} (#2)}

% Integral dx
\newcommand{\dx}{\mathrm{d}x}

% Alias for the Solution section header
\newcommand{\solution}{\textbf{\large Solution}}

% Probability commands: Expectation, Variance, Covariance, Bias
\newcommand{\E}{\mathrm{E}}
\newcommand{\Var}{\mathrm{Var}}
\newcommand{\Cov}{\mathrm{Cov}}
\newcommand{\Bias}{\mathrm{Bias}}

\begin{document}

\maketitle

\pagebreak

\begin{homeworkProblem}
    Give a definition of mathematical function.\\

    \textbf{Solution}\\
    In mathematics, a function is a relation from a set of inputs to a set of possible outputs.
    A function maps a single input value (or argument) within the input domain to a single 
    output value falling within the function range.

\end{homeworkProblem}

\begin{homeworkProblem}
    What ways of defining mathematical functions do you know?\\


    \textbf{Solution}\\
    Three ways:
    \begin{itemize}
        \item Implicit functions - a function that will equate to 0\\
        e.g.: \(f(x,y,z,t) = 0\), where \(x,y,z\) are Cartesian 
        coordinates and \(t\) is the time

        \item Explicit functions - a function which output is manipulated by its inputs\\
        e.g.: \(g=f(x,y,z,t)\)

        \item Parametric functions - a set of explicit functions which creates values 
        of co-ordinates from inputs consisting of variables from other domain
         (i.e., parametric coordinates)\\
        e.g.: \(x = f_x(y,v,w,t); y=f_y(u,v,w,t); z=f_z(u,v,w,t)\)
        

    \end{itemize}
    
\end{homeworkProblem}

\begin{homeworkProblem}
    Given an explicit function \(y = sin(x) + cos(x)\), propose how to convert it to the
    respective parametric functions \(x = f_1(t)\) \(y = f_2(t)\)?\\

    \textbf{Solution}\\
    We can define the first parametric function as:\\
    \(\mathbf{x = f_1(t) = t}\)\\
    The second parametric function can then be obtained as:\\
    \(\mathbf{y = f_2(t) = sin(t) + cos(t)}\)
    


\end{homeworkProblem}

\pagebreak
\begin{homeworkProblem}
    \begin{enumerate}[i]
        \item Given parametric functions \(x = sin^2(t)\) and \(y = cos(t)\), obtain the
        respective implicit function \(f(x,y) = 0\).
        \item Given parametric functions \(x = 2 + 3t\) and \(y = 3 + t\), obtain the
        respective implicit function \(f(x,y) = 0\).
    \end{enumerate}

    \textbf{Solution}\\
    \textbf{Part i}\\
    \[
        \begin{split}
            y  &= cos(t)
            \\
            y^2 &= cos^2(t)
            \\ 
            \therefore x + y^2 &= sin^2(t) + cos^2(t)
            \\
            &= 1
            \\
            \mathbf{f(x,y) = x + y^2 - 1} & \mathbf{\ = 0}
        \end{split}
    \]
    \\\\
    \textbf{Part ii}\\
    \[
        \begin{split}
            y  &= 3+t
            \\
            3y &= 9+3t
            \\ 
            3t &= 3y - 9
            \\
            &
            \\
            x &= 2 + (3y - 9)
            \\
            &= 3y - 7
            \\
            x - 3y &= -7
            \\
            \mathbf{f(x,y) = x - 3y + 7} &\mathbf{\ = 0}
        \end{split}
    \]


    
\end{homeworkProblem}

    



\end{document}